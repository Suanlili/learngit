\documentclass{ctexart}

\usepackage{graphicx}
\usepackage{amsmath}

\title{作业二}


\author{张立言 \\ 数学与应用数学 3210101027}

\begin{document}

\section{学习报告}
这篇报告基于Beginning Linux Programming 40-60页内容。主要报告了trap命令的学习心得。

\subsection{关于trap命令的学习体会}
trap命令可以将接受到的指令转化为一定的动作。通常,trap的用途是当代码运行时如果收到外部的干扰,可以更加整洁。

trap命令的格式是:

\textbf{trap}{} \textit{command signal}

signal是代码运行时会接受到的信号,command是代码接受到了指令之后进行的动作。通常我们将trap放在我们想要保护的代码之前。

让trap命令初始化我们只需要把command部分写成-。而为了忽略某一个指令,我们可以将command设置为空字符''。

signal部分的信号我们可以参阅\verb|signal manual|中列出来的所有信号。我们可以利用一些常用的信号来测试所学。比如INT信号,通常是由按下Ctrl+C来发出,还有HUP,QUIT等信号。

\subsection{程序实验}
执行以下的代码:
\begin{verbatim}
#!/bin/bash

trap 'rm -f ./temp/my_tmp_file' INT
#INT信号输入时,我们执行删除my_tmp_file的指令,使得下面的循环终止
echo creating file/tmp/my_tmy_file

date > ./temp/my_tmp_file
#将时间信息写入文件

echo "press interrupt(Ctrl+C) to interrupt ..."
while [ -f ./temp/my_tmp_file ]; do
    echo File exists
    sleep 1
done
#如果用户不按下Ctrl+C,这个循环就不会停止

exit 0
\end{verbatim}
运行结果如下:
\begin{verbatim}
creating file/tmp/my_tmy_file
press interrupt to interrupt ...
File exists
File exists
File exists
File exists
^CEscaped!
\end{verbatim}
可以看到,按下Ctrl+C的时候跳出了循环。

对比以下的代码:
\begin{verbatim}
#!/bin/bash

trap 'rm -f ./temp/my_tmp_file' INT
trap QUIT
#QUIT信号不会反映
echo creating file/tmp/my_tmy_file
date > ./temp/my_tmp_file

echo "press interrupt to interrupt ..."
while [ -f ./temp/my_tmp_file ]; do
    echo File exists
    sleep 1
done
echo Escaped!

exit 0
\end{verbatim}
运行后结果如下:
\begin{verbatim}
creating file/tmp/my_tmy_file
press interrupt to interrupt ...
File exists
File exists
File exists
File exists
File exists
File exists
^\./new: line 12:  2402 Quit                    (core dumped) sleep 1
File exists
File exists
File exists
^\./new: line 12:  2406 Quit                    (core dumped) sleep 1
File exists
File exists
File exists
^CEscaped!
\end{verbatim}
运行结果中,有两行是测试的时候按下Ctrl+\ 按钮,在trap中它表示不执行任何动作。可见,只有按下Ctrl+C才能跳出循环。但是Ctrl+Q却不能跳出循环。

\section{学到的内容}
这次阅读使我学到了很多linux系统shell programming的很多知识。

shell programming与我之前所学到的C语言和Python等内容有相同之处。像while, case等的功能与C语言十分类似。但是也有很多不同之处,最主要的区别就是用法的不同。而且,作为操作系统的指令,shell programming的指令更突出了与文件的交互,而不去做数值的编程。

一个令我印象很深的运算符是Dot Command,它使得同一个shell里拥有不同的环境,比$\#{}$include更加方便。我认为它能很好体现shell编程的特点,它能够极大地方便计算机使用者,充分体现了shell的优势。

trap命令是一个非常有用的命令,它有利于代码的优化,同时也让程序变得容易控制,可以为意外事件提供一定的解决方法。



\end{document}
