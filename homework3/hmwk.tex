\documentclass{ctexart}

\bibliographystyle{plain}

\usepackage{graphicx}
\usepackage{amsmath}
\usepackage{url}

\title{作业三}


\author{张立言 \\ 数学与应用数学 3210101027}

\begin{document}

这是关于我的Linux工作环境的一个小报告。

\section{我的Linux系统版本信息}
利用查询版本信息的代码\cite{CSDNLinuxcommand},在我的电脑下运行如下:
\begin{verbatim}
Distributor ID:	Ubuntu
Description:	Ubuntu 16.04 LTS
Release:	16.04
Codename:	xenial
\end{verbatim}
Linux发行版本名称为Ubuntu,版本号为16.04。

\section{我的配置工作}

在虚拟机安装好之后,我首先设置了中文输入法。随后,我在Synaptic Package Manager中安装了必备的一些工作环境。例如g++,ssh,git,以及用来编写latex的texlive等,安装了emacs以及doxymacs和doxygen等。最后,我修改了部分emacs的配置文件,加入了中文输入。

\section{下一步工作的规划}

\subsection{使用Linux环境的场合}

在接下来的工作中,我可以进一步熟悉Linux环境,用来提高工作的效率。未来半年内,我可以利用latex完成很多计算机和数学方面的作业。利用shell programming,可以编写一些提高管理效率的脚本来方便我管理自己的计算机,节约时间和资源。

\subsection{分析}

Linux环境非常有利于高效率的工作,在近几天的接触学习之下,我认为我对于Linux掌握尚且不熟练,应该进一步适应和掌握使用Linux环境。我的计划是未来多去使用Linux进行工作和学习,在掌握Linux的基础上再根据自己的需求作出调整和改变,以适应我未来更多的工作。

\section{如何保证我的资料的安全?}

我学习了如何使用git来托管我的代码\cite{2020Introduction}。现在,我已经在我的虚拟机里创建了一个git的库并且把最近的作业放在里面,方便我了解和学习git。在进一步掌握git之后,我会使用github或者其他的代码托管网站来托管我未来的工作和学习资料。这是保护我的代码安全的一个十分有效的手段。


\bibliography{try}
\end{document}
