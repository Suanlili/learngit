\documentclass{ctexart}
\bibliographystyle{plain}

\usepackage{graphicx}
\usepackage{amsmath}
\usepackage{amsfonts,amssymb}

\begin{document}

\title{作业4\\Mandelbrot Set的生成和探索}
\author{张立言\\数学与应用数学 3210101207}
\maketitle

\begin{abstract}
Mandelbrot set是十分有趣的数学现象,它凭借神奇的图像特征而变得广为人知。本报告利用了C++和Bitmap库生成Mandelbrot set的图像并作出一些处理和探索,并且引申了tricorn set的图像。

\end{abstract}
\section{引言}
Mandelbrot set,中文名又叫做曼德博集合。它是由复平面上组成分形的点组成的一种集合,最早在1978年由Robert W.Brooks和Peter Matelski定义并提出。后来在1980年由Benoit Mandelbrot作出了可视化处理并且最终广为人知。它的神奇之处在于,在图形的“边界”处不断进行放大总会有更多的细节显示出来。关于它的一些性质,许多数学家作出了探索,并且得出了一些结论,而也有很多猜想尚未得到证明,例如其局部联通性质(local connectedness)和某些特殊点的自相似性(self-similarity)。我们利用一些简单的数学结论辅助完成本次可视化Mandelbrot set的工作。

\section{问题的背景介绍}
我们不必要关心Mandelbrot set的拓扑性质,但可以利用一些简单的结论辅助我们优化我们的可视化处理。绘制Mandelbrot set的图像的主要目的是练习掌握和使用Linux系统工作的技巧。
\section{数学理论}
Mandelbrot set的通常定义如下:
使得复平面上的迭代方程
\begin{equation}
z_{n+1}=z_n^2+c \tag{iteq}
\end{equation}
在初始值$z_0=0$时进行迭代能够最终收敛的所有$c$的集合。其中,$c \in \mathbb{C}$.
下面给出一些关于该集合的一些基本的性质:
在实轴上,Mandelbrot set为严格的闭区间$[-2,\frac{1}{4}]$.拓扑学家证明了它的连通性。接下来介绍一些对于我们编写程序很有用的结论:
\subsection{Theorem 1}
\textbf{若$c\in M$,则$|c|\le 2$}

\textit{Proof}{} 假设$|c|>2$,则$|z_1|=|c|,|z_1|>2$\par
当$n=2$时,
$$|z_2|=|z_1^2+c|\le |c^2|-|c|$$
由$|c|>2$可知
$$|c|^2-|c|>|c|$$
从而$|z_2|>|c|$\par
假设$|z_n|>|c|$成立,则$|z_n|>2$,
$$|z_{n+1}|=|z_n^2+c|\le |z_n|^2-|c|$$
又因为$|z_n|>2$,从而
$$|z_{n+1}|>|z_n|^2-|z_n|>|z_n|$$
可知$|z_n|$递增,从而$|z_n|>|z_1|>2$.假如$|z_n|$不发散,由于它递增,从而收敛至某一常数$a$.于是由$|z_{n+1}|\le |z_n|^2-|c|$取极限可得$a\le a^2-|c|\Rightarrow a^2-a=a(a-1)\le a\le |c|$,矛盾,故而$|z_n|$发散。
\subsection{Theorem 2}
\textbf{若$c \in M$则$|z_n|\le 2$}



\end{document}
