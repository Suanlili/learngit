\documentclass{ctexart}

\usepackage{amsmath}
\usepackage{graphicx}

\begin{document}
\title{作业五}
\author{张立言\\数学与应用数学\\3210101207}
\maketitle

\section{roots.c程序的分析}
这是一个求解方程$x^2-5=0$的程序。利用了迭代法,设置了最大迭代次数为100次并且调用了gsl提供的函数进行迭代。设置了解的范围为$[1,5]$,以迭代下界作为方程的近似解。并且输出了多次迭代的结果与精确解$\sqrt{5}$的误差err,最终利用gsl提供的函数判断收敛。从输出结果可以看到,利用gsl提供的函数,只需迭代6次,就能很接近精确解。这体现了gsl的强大之处。

\end{document}
